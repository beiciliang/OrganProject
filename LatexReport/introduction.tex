
\chapter{Introduction}

\section{Motivation}

This project is a five months advanced placement as part of the Organ Project undertaken at the Union Chapel in the summer of 2015. The main task of the project is to reach out beyond the chapel's live audiences by creating an app on a chosen platform that features a small collection of audio samples from the Henry Willis pipe organ, which can be used in a creative interactive way to the user.

The organ at Union Chapel was designed and built specially for the size and acoustics of the new chapel building in 1877 by master organ builder Henry \enquote{Father} Willis. It is undoubtedly one of the finest in the world. It is one of just two organs left in the United Kingdom, and the only one in England, with a fully working original hydraulic (water powered) blowing system, which can be used as an alternative to the electric blowers. The organ is deliberately hidden away behind ornate screens under the rose window as shown in Figure~\ref{fig:interior} to make the congregation avoid being distracted by the sight of an organ or organist so as to let the music itself to be the focus during worship, which itself actually hints at the organ's importance, with its depiction of eight angels all playing different musical instruments.

\begin{figure}
\centerline{\epsfig{figure=images/interior.jpg,width=0.8\textwidth}}
\caption{The organ at the Union Chapel.} 
\label{fig:interior}
\end{figure}

As the Organ Project aims not only to honour the legacy of this very special instrument by keeping up a regular organ recital diary and having it performed in styles for the live audiences, but also to make it more accessible beyond the local community. Also since now HTML5 can embed audio and video content directly into the web page, which brings more possibilities for the web multimedia technology, typically leading to expand audiences having access to more diverse options. The investigation presented in this report outlines the development of a web application that is designed as an exploratory tool to provide the user with information about different parts of the Henry Willis pipe organ, as well as to play the organ with six different timbres using computer keyboard and mouse.

\section{Research questions}

When researching on how to make the Henry Willis pipe organ more accessible not only for the live audiences, but also for the people who don't have access to the Union Chapel, I was faced with the following research questions:

\begin{enumerate}
\item What are the platform we can develop the app on? Web app or native app?
\item How can we design the interface based on a chosen platform?
\item What are the implementation steps that should be taken to build the app in order to enhance the accessibility of the organ?
\end{enumerate}

\section{Summary of the structure of the report}

Firstly the report begins with a discussion about how digital technologies take effect on music applications. I shall contextualise the work by looking at how the ways of making, sharing, teaching and learning music has been changed through the digital technologies. My intention is to highlight the importance of web application design and web technology. With a particular focus on the web application I shall then develop a framework for thinking about the development challenges of this project.

Following this is a section which provides the detailed design and implementation process, including any requirement the app need to meet, any specification compared with the native app, and implementation guidelines, highlighting the main API and libraries applied in the app.

The next section presents the evaluation of the app by online questionnaire. Strong and weak aspects of the usability are identified, as well as aspects that would benefit from user evaluation.

The final section presents the future work and conclusion. All in all, the report aims to demonstrate how versatile the organ can be through the web-based app - that it is not simply a church instrument, but in fact the world?s first synthesiser which can be incorporated into most genres of music.
